%Copyright 2014 Jean-Philippe Eisenbarth
%This program is free software: you can
%redistribute it and/or modify it under the terms of the GNU General Public
%License as published by the Free Software Foundation, either version 3 of the
%License, or (at your option) any later version.
%This program is distributed in the hope that it will be useful,but WITHOUT ANY
%WARRANTY; without even the implied warranty of MERCHANTABILITY or FITNESS FOR A
%PARTICULAR PURPOSE. See the GNU General Public License for more details.
%You should have received a copy of the GNU General Public License along with
%this program.  If not, see <http://www.gnu.org/licenses/>.

%Based on the code of Yiannis Lazarides
%http://tex.stackexchange.com/questions/42602/software-requirements-specification-with-latex
%http://tex.stackexchange.com/users/963/yiannis-lazarides
%Also based on the template of Karl E. Wiegers
%http://www.se.rit.edu/~emad/teaching/slides/srs_template_sep14.pdf
%http://karlwiegers.com
\documentclass{scrreprt}
\usepackage{listings}
\usepackage{underscore}
\usepackage[bookmarks=true]{hyperref}
\usepackage[utf8]{inputenc}
\usepackage[english]{babel}
\usepackage{graphicx}
\usepackage{hyperref}
\newcommand\tab[1][0.5cm]{\hspace*{#1}}
\graphicspath{ {images/} }
\hypersetup{
    bookmarks=false,    % show bookmarks bar?
    pdftitle={Software Requirement Specification},    % title
    pdfauthor={OSS},                     % author
    pdfsubject={TeX and LaTeX},                        % subject of the document
    pdfkeywords={TeX, LaTeX, graphics, images}, % list of keywords
    colorlinks=true,       % false: boxed links; true: colored links
    linkcolor=blue,       % color of internal links
    citecolor=black,       % color of links to bibliography
    filecolor=black,        % color of file links
    urlcolor=purple,        % color of external links
    linktoc=page            % only page is linked
}%
\def\myversion{1.0 }
\date{}
%\title
\begin{document}

\begin{flushright}
    \rule{16cm}{5pt}\vskip1cm
    \begin{bfseries}
        \Huge{SOFTWARE REQUIREMENTS\\ SPECIFICATION}\\
        \vspace{1.9cm}
        for\\
        \vspace{1.9cm}
        Mattermost Adoption Project\\
        \vspace{1.9cm}
        \LARGE{Version \myversion}\\
        \vspace{1.9cm}
        Prepared by Open System Solutions\\
        \vspace{1.9cm}
        \today\\
    \end{bfseries}
\end{flushright}

\tableofcontents


%\chapter*{Revision History}
%
%\begin{center}
%    \begin{tabular}{|c|c|c|c|}
%        \hline
%	    Name & Date & Reason For Changes & Version\\
%        \hline
%	    21 & 22 & 23 & 24\\
%        \hline
%	    31 & 32 & 33 & 34\\
%        \hline
%    \end{tabular}
%\end{center}

\chapter{Introduction}

\section{Purpose}
Mattermost is an open-source SaaS messaging service used
by several enterprise level organizations such as Samsung, Urban Outfitters,
Red Hat, Mozilla, The University of British Columbia, Anthropologie, analyticsMD
and LegalIO.\\

The following document is meant to detail the features of Mattermost as well as
the necessary software and hardware needed to implement these features.

\section{Intended Audience}
The primary audience is the Office of Information Technology as well as members
of Open System Solutions.\\

Specifically, this document is to be read by the development team,
project managers, stake holders, and testers.

\section{Project Scope}
%Mattermost is intended to replace the informal use of Slack within University departments.  As stated in Section ABCXYZ, ``Thou shalt not... `` existing messaging service systems and chat
%clients with a new web-based system with features detailed within this
%document.\\
We plan on setting up Mattermost for the purpose of consolidating messaging for
multiple departments without access to jabber such as Rutgers Biomedical
and Health Sciences (RBHS).\\

The scope of this project includes deploying a local Mattermost server within
Rutgers University and progressively rolling out this service to departments
within the University.  Additionally, we can service any customization requests
for the application as it is open-source.\\

Additionally, alternative messaging services such as Slack or Office 365 do
not currently meet our expectations for the following reasons:
\begin{itemize}
  \item Slack: Slack currently stores message information on their remote
    cloud systems.  This violates Section B3 of the \textbf{Rutgers University Acceptable
    Use Policy for Information Technology Resource} which states:\\
    \noindent
    \textit{Using third party, cloud and non-cloud, systems not authorized or 
      approved by OIT’s Information Protection and Security (IPS) Division to 
      transmit, process, store Rutgers data classified as restricted, including 
      ePHI. Contact OIT’s IPS division for additional information on using 
      third party systems.}
  \item Office 365: Office 365 does not meet our groups requirements due to
    lacking source control and unit testing integration.
\end{itemize}

The long-term goal is to introduce this service to interested departments and
customize the pre-existing application to the unique needs of Rutgers
departments.

\section{References}
%\textbf{Acceptable Use Policy for Information Technology Resources Policy:}\\
%http://policies.rutgers.edu/sites/policies/files/00005638.PDF\\

\noindent
\textbf{Mattermost Features List:}\\
https://www.mattermost.org/features/\\

\noindent
\textbf{Mattermost Integration Apps:}\\
https://www.mattermost.org/community-applications/\\


\chapter{Overall Description}

\section{Product Perspective}
%We plan to use Mattermost as a centralized messaging solution which is easy to
%manage and low cost for the Office of Information Technology Department (OIT)
%at Rutgers University.

%Currently, many OIT Employees use messaging services
%which store chat information either locally on their machines (e.g. Pidgin) or
%in remote servers inaccessible to Rutgers Systems Administrators (e.g. Office
%365) which makes retrieving the information upon request difficult.\\

We plan to use Mattermost as a centralized messaging solution which is easy to
manage and low cost for departments which lack a messaging platform (e.g.
Pidgin) such as RBHS as mentioned in the Project Scope.

Additionally, Setting up a messaging system with Mattermost within Rutgers allows the
University to easily maintain employee chat information as well as provide
features specific to the needs of Rutgers employees with its modifiable code
base.\\

To support the OIT department with Mattermost, we would require a single server
with 4 vCPUs/cores, 8 GB RAM, and 90-180 GB storage.\\

The following image is a diagram of the major components of the overall system:
\includegraphics[scale=0.5, width=\textwidth]{mattermost-diagram.png}

\section{Product Functions}
Mattermost can provide the following features:\\
\tab 1. A website, android app and iOS app to access your Mattermost
account, allowing \tab employees to pick up conversations where they left off when
leaving the desk\\
\tab 2. Share files and images within chats\\
\tab 3. Attach sound, video and image files from mobile devices\\
\tab 4. Unlimited search history (Slack is limited to 10000 lines/chat)\\
\tab 5. Search public and private channels for historical messages and
comments with Discovery in mind\\
\tab 6. CAS support\\
\tab 7. Open Source integrations service providing Gitlab and other software
support\\
\tab 8. The ability to create Rutgers specific addons (e.g. JIRA integration) since
it is \tab opened source\\
\tab 9. Define team-specific branding and color themes across your devices
\tab 10. Host and manage dozens of teams from a single server instance\\
\tab 11. System Console and command line tools for managing essential IT
administration \tab functions\\
\tab 12. Features which Slack does not offer:\\
\tab\tab a) Comment threads – to circle back on any topic\\
\tab\tab b) Non-English channel names – to support multi-lingual teams\\
\tab\tab c) Hashtags – to quickly tag and find any message\\
\tab\tab d) Auto-highlight messages you wrote – to easily see what you said\\

In the long term, we see our Mattermost deployment being able to serve any group
of Rutgers employees seeking a free and customizable messaging service.

\section{User Classes and Characteristics}
\textbf{Rutgers Employee:}
\tab The primary users are expected to be Rutgers employees who will use this
service to communicate amongst one another.  As knowledgeable users, we
anticipate to receive feature requests from them.\\

\noindent
\textbf{System Administrator:}
\tab The System Administrator has the priviledge to update and retrieve
information in the database.

\section{Operating Environment}
The Mattermost client application can be installed on machines with operating
systems of:
\begin{itemize}
\item Windows 7+ (IE 11*, Chrome 43+, Firefox 38+, and Edge)
\item Mac 10.9+ (Safari 9, Chrome 43+)
\item Linux - Arch 4.0.0 (Chrome 43+)
\item Android 4.4+
\item iOS 9+
\end{itemize}

\section{Design and Implementation Constraints}
%$<$Describe any items or issues that will limit the options available to the
%developers. These might include: corporate or regulatory policies; hardware
%limitations (timing requirements, memory requirements); interfaces to other
%applications; specific technologies, tools, and databases to be used; parallel
%operations; language requirements; communications protocols; security
%considerations; design conventions or programming standards (for example, if the
%customer’s organization will be responsible for maintaining the delivered
%software).$>$

\section{User Documentation}
User documentation for Mattermost can be found in the following site:\\
https://docs.mattermost.com/guides/user.html.

%\section{Assumptions and Dependencies}
%$<$List any assumed factors (as opposed to known facts) that could affect the
%requirements stated in the SRS. These could include third-party or commercial
%components that you plan to use, issues around the development or operating
%environment, or constraints. The project could be affected if these assumptions
%are incorrect, are not shared, or change. Also identify any dependencies the
%project has on external factors, such as software components that you intend to
%reuse from another project, unless they are already documented elsewhere (for
%example, in the vision and scope document or the project plan).$>$


%\chapter{External Interface Requirements}
%
%\section{User Interfaces}
%$<$Describe the logical characteristics of each interface between the software
%product and the users. This may include sample screen images, any GUI standards
%or product family style guides that are to be followed, screen layout
%constraints, standard buttons and functions (e.g., help) that will appear on
%every screen, keyboard shortcuts, error message display standards, and so on.
%Define the software components for which a user interface is needed. Details of
%the user interface design should be documented in a separate user interface
%specification.$>$
%
%\section{Hardware Interfaces}
%$<$Describe the logical and physical characteristics of each interface between
%the software product and the hardware components of the system. This may include
%the supported device types, the nature of the data and control interactions
%between the software and the hardware, and communication protocols to be
%used.$>$
%
%\section{Software Interfaces}
%$<$Describe the connections between this product and other specific software
%components (name and version), including databases, operating systems, tools,
%libraries, and integrated commercial components. Identify the data items or
%messages coming into the system and going out and describe the purpose of each.
%Describe the services needed and the nature of communications. Refer to
%documents that describe detailed application programming interface protocols.
%Identify data that will be shared across software components. If the data
%sharing mechanism must be implemented in a specific way (for example, use of a
%global data area in a multitasking operating system), specify this as an
%implementation constraint.$>$
%
%\section{Communications Interfaces}
%$<$Describe the requirements associated with any communications functions
%required by this product, including e-mail, web browser, network server
%communications protocols, electronic forms, and so on. Define any pertinent
%message formatting. Identify any communication standards that will be used, such
%as FTP or HTTP. Specify any communication security or encryption issues, data
%transfer rates, and synchronization mechanisms.$>$


\chapter{System Features}

\section{Message History Synchronization over mulitple devices}
\subsection{Description}
In order to be able to synchronize message history between devices using
Mattermost, we must setup a Mattermost server which will manage all of the
message history from a centralized server.
\subsection{Priority}
This requirement is HIGH priority.
\subsection{Functional Requirements}
To create this centralized server, we will first need to obtain the hardware
specified in the Product Perspective.  We will be installing our Mattermost
server in a CentOS 7 operating system.
First, we will setup the
database server (PostgreSQL 9.4+/MySQL 5.6+) to hold all of the users' chat
information.  Once this is completed, we will setup proper user credentials on
our machine along with an nginx server to serve message information to client
devices.  A more detailed outline of installation steps can be found here:
https://docs.mattermost.com/install/prod-rhel-7.html.

\section{CAS Support}
To Be Determined.

\section{Github Integration}
\subsection{Description}
Github integration would provide a mattermost channel for github messages for
teams to keep track of like so:\\
\includegraphics[scale=0.5, width=\textwidth]{github-integration.png}

Github integration is one of many examples of possible additions that can be
made to our Mattermost server.  Mattermost uses webhooks to receive real-time
data from third party applications.
\subsection{Priority}
This requirement is LOW priority.
\subsection{Functional Requirements}
This integration would require us to run a python server requiring the flask
and requests python packages.  Further details for installtion can be found
here: https://github.com/softdevteam/mattermost-github-integration

\section{Theme Customization}
\subsection{Description}
Mattermost allows us to customize the theme of the application, allowing us to
set the theme to match the Rutgers colors or colors specified by a department.
\subsection{Priority}
This requirement is LOW priority.
\subsection{Functional Requirements}
The theme colors can be manipulated through the ``Theme'' Setting which can be
accessed through the ``Account Setting'' \textgreater ``Display'' \textgreater ``Theme'' menu
options.  More information can be found here:
https://docs.mattermost.com/help/settings/theme-colors.html


%\chapter{Other Nonfunctional Requirements}
%
%\section{Performance Requirements}
%$<$If there are performance requirements for the product under various
%circumstances, state them here and explain their rationale, to help the
%developers understand the intent and make suitable design choices. Specify the
%timing relationships for real time systems. Make such requirements as specific
%as possible. You may need to state performance requirements for individual
%functional requirements or features.$>$
%
%\section{Safety Requirements}
%$<$Specify those requirements that are concerned with possible loss, damage, or
%harm that could result from the use of the product. Define any safeguards or
%actions that must be taken, as well as actions that must be prevented. Refer to
%any external policies or regulations that state safety issues that affect the
%product’s design or use. Define any safety certifications that must be
%satisfied.$>$
%
%\section{Security Requirements}
%$<$Specify any requirements regarding security or privacy issues surrounding use
%of the product or protection of the data used or created by the product. Define
%any user identity authentication requirements. Refer to any external policies or
%regulations containing security issues that affect the product. Define any
%security or privacy certifications that must be satisfied.$>$
%
%\section{Software Quality Attributes}
%$<$Specify any additional quality characteristics for the product that will be
%important to either the customers or the developers. Some to consider are:
%adaptability, availability, correctness, flexibility, interoperability,
%maintainability, portability, reliability, reusability, robustness, testability,
%and usability. Write these to be specific, quantitative, and verifiable when
%possible. At the least, clarify the relative preferences for various attributes,
%such as ease of use over ease of learning.$>$
%
%\section{Business Rules}
%$<$List any operating principles about the product, such as which individuals or
%roles can perform which functions under specific circumstances. These are not
%functional requirements in themselves, but they may imply certain functional
%requirements to enforce the rules.$>$
%
%
%\chapter{Other Requirements}
%$<$Define any other requirements not covered elsewhere in the SRS. This might
%include database requirements, internationalization requirements, legal
%requirements, reuse objectives for the project, and so on. Add any new sections
%that are pertinent to the project.$>$
%
%\section{Appendix A: Glossary}
%%see https://en.wikibooks.org/wiki/LaTeX/Glossary
%$<$Define all the terms necessary to properly interpret the SRS, including
%acronyms and abbreviations. You may wish to build a separate glossary that spans
%multiple projects or the entire organization, and just include terms specific to
%a single project in each SRS.$>$
%
%\section{Appendix B: Analysis Models}
%$<$Optionally, include any pertinent analysis models, such as data flow
%diagrams, class diagrams, state-transition diagrams, or entity-relationship
%diagrams.$>$
%
%\section{Appendix C: To Be Determined List}
%$<$Collect a numbered list of the TBD (to be determined) references that remain
%in the SRS so they can be tracked to closure.$>$

\end{document}
